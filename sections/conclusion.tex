\section{Conclusion}
This work demonstrates that deadband configurations play a large role in dictating valve travel.
Counterintuitively, a step deadband can lead to vastly increased valve travel compared to other, more linear, options.
Smaller deadbands can reduce valve travel and help regulate frequency if adopted interconnection-wide.
In a system with various deadband settings, machines with smaller deadbands will respond more than machines with larger deadbands.

Further, this work demonstrates the need for a simulation environment that can capture long-term power system dynamics with appropriate static and dynamic models. 
The TSPF approach to long term dynamic simulation is shown be be most useful.
Future work in this area will focus on using the cumulative valve travel metric to study and optimize various governor and AGC control strategies.

%However, in a system with a variety of deadbands, machines with smaller deadbands will respond more than machines with larger deadbands.
%This creates an uneven distribution of valve travel in a system.
%Therefore, if smaller deadbands are to be applied, they should be done interconnection wide to avoid unintentional inequalities in valve travel and associated machine wear.