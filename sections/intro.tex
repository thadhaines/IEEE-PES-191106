\section{Introduction}
Deadbands are commonly implemented in turbine speed governors to prevent a machine from responding to small frequency deviations that are ever present in electrical systems.
Industry usage of deadbands is widely known, however, it is often overlooked in power system simulation.
Incorporating governor deadbands into transient simulation models has been shown to generate results that better match measured power system events\cite{kou2016}.
Additionally, numerous long-term operational benefits have been attributed to the alteration of deadband behavior\cite{nercFRI2012}.
To better understand how deadbands may affect system behavior, a long-term simulation would seem suitable.
Long-term dynamic (LTD) simulation software focusing on governor dynamics was proposed in \cite{AGCCresap} and others, but has not been standardized nor fully developed.


This paper explores the impact of governor deadbands using a novel LTD simulation environment based on GE Energy's Positive Sequence Load Flow (PSLF). 
Valve travel as a metric for comparing the impact of various deadband scenarios is proposed. 
Section II describes the LTD simulation environment used to perform the studies. 
Section III discusses governor and deadband modeling within the simulation environment. 
In Section IV, the LTD simulation is validated against mature transient stability simulation. 
Finally, Section V presents some initial results demonstrating the viability of the proposed valve travel metric and Section VI offers conclusions.
