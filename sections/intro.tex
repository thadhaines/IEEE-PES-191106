\section{Introduction}
Deadbands are commonly implemented in turbine speed governors to prevent a machine from responding to small frequency deviations that are ever present in electrical systems. Recently, the North American Electric Reliability Corporation (NERC) undertook an initiative to study the degradation of primary frequency response in the eastern interconnection. The initiative culminated in NERC Industry Advisory A-2015-02-05-01\cite{NERCxxxx} recommending a maximum deadband setting of $\pm 0.036Hz$ for generators intended to supply primary frequency response. The recommendation was later adopted by the Federal Energy Regulatory Commission (FERC) in FERC Order 842, which mandates that utility pro forma Interconnection Agreements include the deadband limitation\cite{FERCxxxx}.

The NERC advisory, and the body of work leading to the advisory, also document a lack of accurate modeling related to governor deadbands, and commensurate accuracy issues when using classical transient stability models for studying primary frequency response issues. Industry usage of deadbands is widely known, however, it is often overlooked in power system simulation.
\cite{kou2016} details how incorporating governor deadbands into transient simulation models can lead to results that better match measured power system events. [TODO: clean this up... \cite{nercFRI2012} reports frequency profile improvement caused by no step deadband. Also uses MW-min calculation to state smaller no step deadbands reduce movement, calls for immediate dev of gov guidelines - led to standard TRE BAL-001 for ERCOT - in full effect april 2015. Most Recent Report \cite{nercFRAA2018} has odd ERCOT frequency probability - more U shaped, less normal.]

Transient stability simulation is useful for the study of primary frequency response, however there is little understanding of the impact of governor deadbands on valve travel of individual generating units. Do deadbands really reduce overall valve travel? By how much? Does a deadband on one unit increase overall valve travel of other units within the Balancing Authority (BA)? Outside of the BA? It is difficult to answer these questions within the framework of a classical transient stability simulation due to the relatively small integration timestep. These questions require the development of a simulation environment capable of simulating tens of minutes to hours.

Cresap and others proposed solving the equations of motion for a single swing equation incorporating the entire aggregated system inertia in \cite{AGCCresap} and \cite{CarsonTaylor}. Using this concept, the aggregate system frequency response can be reasonably captured. In this work, the authors take the concept further to enable the study of governor deadbands over a much simulation duration than can be reasonably accomplished with a classical transient stability simulation. The paper explores the impact of governor deadbands using a novel long term simulation environment based on GE Energy's Positive Sequence Load Flow (PSLF)\cite{PSLFUserManual}. 
The paper proposes a cumulative valve travel metric for comparing the impact of various deadband scenarios. 
Section II describes the long term simulation environment used to perform the studies. 
Section III discusses governor and deadband modeling within the simulation environment. 
In Section IV, the long term simulation is validated against mature transient stability simulations. 
Finally, Section V presents some initial results demonstrating the viability of the proposed valve travel metric and Section VI offers conclusions.


