\section{Introduction}
Deadbands are commonly implemented in turbine speed governors to prevent a machine from responding to small frequency deviations that are ever present in electrical systems.
Industry usage of deadbands is widely known, however, it is often overlooked in power system simulation.
\cite{kou2016} details how incorporating governor deadbands into transient simulation models can lead to results that better match measured power system events.


mish mash intro thoughts:
\begin{itemize}
	\item deadband simulation considered in transient stability time frame, not long-term
	\item \cite{nercFRI2012} reports frequency profile improvement caused by no step deadband. Also uses MW-min calculation to state smaller no step deadbands reduce movement, calls for immediate dev of gov guidelines - led to standard TRE BAL-001 for ERCOT - in full effect april 2015. Most Recent Report \cite{nercFRAA2018} has odd ERCOT frequency probability - more U shaped, less normal.
	\item transient simulation not meant for long term simulation
	\item long term dynamic simulation software focusing on governor dynamics was proposed in [cite Taylor] and others, but has not been standardized nor fully developed.
\end{itemize}


This paper explores the impact of governor deadbands using a novel long term simulation environment based on GE Energy's Positive Sequence Load Flow (PSLF). 
The paper proposes a cumulative valve travel metric for comparing the impact of various deadband scenarios. 
Section II describes the long term simulation environment used to perform the studies. 
Section III discusses governor and deadband modeling within the simulation environment. 
In Section IV, the long term simulation is validated against mature transient stability simulations. 
Finally, Section V presents some initial results demonstrating the viability of the proposed valve travel metric and Section VI offers conclusions.


