\section{Introduction}
%Deadbands are commonly implemented in turbine speed governors to prevent a machine from responding to small frequency deviations that are ever present in electrical systems.
%Industry usage of deadbands is widely known, however, it is often overlooked in power system simulation.
%Incorporating governor deadbands into transient simulation models has been shown to generate results that better match measured power system events\cite{kou2016}.
%Additionally, numerous long-term operational benefits have been attributed to the alteration of deadband behavior\cite{nercFRI2012}.
%To better understand how deadbands may affect system behavior, a long-term simulation would seem suitable.
%Long-term dynamic (LTD) simulation software focusing on governor dynamics was proposed in \cite{AGCCresap} and others, but has not been standardized nor fully developed.
%
%
% Matt Intro:
%
%Recently, the North American Electric Reliability Corporation (NERC) undertook an initiative to study the degradation of primary frequency response in the eastern interconnection. 
%The initiative culminated in NERC Industry Advisory A-2015-02-05-01\cite{NERCxxxx} recommending a maximum deadband setting of $\pm$36 mHz for generators intended to supply primary frequency response. 
%The recommendation was later adopted by the Federal Energy Regulatory Commission (FERC) in FERC Order 842, which mandates that utility pro forma Interconnection Agreements include the deadband limitation\cite{FERCxxxx}.
%
%The NERC advisory, and the body of work leading to the advisory, also document a lack of accurate modeling related to governor deadbands, and commensurate accuracy issues when using classical transient stability models for studying primary frequency response issues. 
%
%~Industry usage of deadbands is widely known, however, it is often overlooked in power system simulation.
%~ kou ref....
%
%Transient stability simulation is useful for the study of primary frequency response, however there is little understanding of the impact of governor deadbands on valve travel of individual generating units. 
%Do deadbands really reduce overall valve travel? By how much? 
%Does a deadband on one unit increase overall valve travel of other units within the Balancing Authority (BA)? Outside of the BA? 
%It is difficult to answer these questions within the framework of a classical transient stability (CTS) simulation due to the relatively small time frame of focus. 
%These questions require the development of a simulation environment capable of simulating tens of minutes to hours.
%
%Stajcar !Cresap! and others (?) proposed solving the equations of motion for a single swing equation incorporating the entire aggregated system inertia in \cite{AGCCresap} and \cite{CarsonTaylor}. 
%Using this concept, the aggregate system frequency response can be reasonably captured. 
%In this work, the authors take the concept further to enable the study of governor deadbands over a much longer simulation duration than can be reasonably accomplished with classical transient stability simulation. 
%
%% Edited Intro:
Turbine speed governors commonly implement deadbands to prevent a machine from responding to small frequency deviations that are ever present in electrical systems.
% combine previous work of NERC to FERC order
The North American Electric Reliability Corporation (NERC) has shown benefits of modifying governor deadbands in \cite{nercFRI2012}.
The Federal Energy Regulatory Commission (FERC) has adopted NERC deadband recommendations from \cite{nercFRI2012} as FERC Order 842\cite{ferc2018}.
% transient sims showing usefulness of simulation
Industry usage of deadbands is widely known, however, it is often overlooked in power system simulation.
Incorporating governor deadbands into transient simulation models has been shown to generate results that better match measured power system events\cite{kou2016}.
% transient sim not enough to answer long term and mulit-area questions
While transient stability simulation is useful for the study of primary frequency response, there is little understanding of the impact governor deadbands may have in the long-term. 
Due to the time frame of focus involved with classical transient stability (CTS) simulation, the development of a simulation environment capable of simulating tens of minutes to hours is required. 
% previous work done on LTD expanded upon
Long-term dynamic (LTD) simulation software has been proposed in \cite{AGCCresap, Stajcar, DonnellyVoltageControl}, but has yet to be standardized or fully developed.
The use of a single swing equation, as proposed in \cite{AGCCresap} and \cite{Stajcar}, for solving the equations of motion involved with an aggregated system inertia, have been shown to produce an acceptable aggregate system frequency. 
%Using this concept, the aggregate system frequency response can be reasonably captured. 
In this work, the authors take this concept further to enable the study of governor deadbands over a much longer simulation duration than can be reasonably accomplished with CTS simulation. 

% roadmap of paper
This paper explores the impact of governor deadbands using a novel LTD simulation environment based on GE Energy's Positive Sequence Load Flow (PSLF) system models, dynamic data, and load flow solver. 
Valve travel as a metric for comparing the impact of various deadband scenarios is proposed. 
Section II describes the LTD simulation environment used to perform the studies while
Section III discusses governor and deadband modeling within the simulation environment. 
In Section IV, the LTD simulation is validated against mature transient stability simulation. 
Finally, Section V presents some initial results demonstrating the viability of the proposed valve travel metric and Section VI offers conclusions.