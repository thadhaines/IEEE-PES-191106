\section{Simulation Validation}
To validate the chosen simulation approach, identical system perturbations were simulated in both the PSLF Dynamic Subsystem (PSDS), an industry standard CTSS, and PSLTDSim.
%The output data was then compared in MATLAB.
Frequency comparisons were used to validate the single system frequency assumption calculated by PSLTDSim.
Comparisons of governed generator mechanical power were used to validate governor action.

Comparison of frequency data from PSDS to LTD was simplified by calculating a single weighted PSDS frequency $f_w$ based on generator inertia. 
In a system with $N$ generators
\begin{align}
f_{w} &= \sum_{i=1}^{N} f_i \dfrac{H_{PU, i} M_{base, i} }{H_{sys}}  \label{eq: f_weighted}\\
\text{where } H_{sys} &= \sum_{i=1}^{N} H_{PU, i} M_{base, i}.  \label{eq: Hsys}
\end{align}
% This explanation may be unnecessary for IEEE... removed for space
%Due to the different time steps, when calculating the difference between PSDS and LTD, multiple PSDS values have the same held LTD value subtracted from them. 
%For instance, any PSDS($t = 3.x$) value would have the LTD($t = 3$) value subtracted from it for comparison.

% Something about an individual difference not mattering as much as the general difference, therfore all ltd psds comparisons are plotted as a single color.

%-------------------------------------------------------------------------------------
\subsection{The MiniWECC System}

The power system used for validation and valve travel experiments, the `miniWECC' shown in Fig. \ref{fig: miniWECC}, is a 120 bus 34 generator system created in PSLF.
The default miniWECC system was modified to include three areas for non-homogeneous deadband tests.
All 21 governors in the miniWECC were modeled with the deadband modified tgov1 governor.%, which enabled easier validation.
More information about the creation and use of the miniWECC may be found in \cite{trudnowski2012} and \cite{sandia2015}.

\begin{figure}[!ht]
	\centering
	\includegraphics[width=.65\linewidth]{figures/miniWECC_split03}
	\caption{MiniWECC system adapted from \cite{trudnowski2012}.}
	\label{fig: miniWECC}
\end{figure}

%`````````````````````````````````````````````````````````````````````````````````````
\subsubsection{Load Step Results}
A 400 MW load step at $t$ = 2 seconds was used for validating the performance of PSLTDSim.
As shown in Fig. \ref{fig: stepFcomp}, all individual PSDS frequencies begin to oscillate after the perturbance while the weighted PSDS frequency appears to follow the general center of oscillation. The LTD system frequency is less oscillatory than the weighted frequency with only minor differences between the two. Fig. \ref{fig: rampFdif} quantifies these frequency differences.

\begin{figure}[!t]
	\centering
	\includegraphics[width=\linewidth]{figures/miniWECC3ALTDstepF3}
	\caption{Comparison of frequency during load step.}
	\label{fig: stepFcomp}
\end{figure}

\begin{figure}[!t]
	\centering
	\includegraphics[width=\linewidth]{figures/miniWECC3ALTDstepRelF}
	\caption{Absolute difference of weighted frequency during load step.}
	\label{fig: stepFdif}
\end{figure}

When comparing mechanical power in Fig. \ref{fig: stepPmdif}, large MW differences can be seen, however, the percent difference data in Fig. \ref{fig: stepPmPercentdif} shows results less than 5\% max difference, and an average percent difference of less than $\approx$0.5\%.

\begin{figure}[!ht]
	\centering
	\includegraphics[width=\linewidth]{figures/miniWECC3ALTDstepPm2}
	\caption{Difference of mechanical power during load step.}
	\label{fig: stepPmdif}
\end{figure}

\begin{figure}[!ht]
	\centering
	\includegraphics[width=\linewidth]{figures/miniWECC3ALTDstepPm3}
	\caption{Percent difference of mechanical power during load step.}
	\label{fig: stepPmPercentdif}
\end{figure}


%`````````````````````````````````````````````````````````````````````````````````````
\subsubsection{Load Ramp Results}
A second set of validation results, using a 40 second 400 MW load ramp, were also generated. 
Figs. \ref{fig: rampFcomp}-\ref{fig: rampFdif} show frequency of LTD being within 1.5 mHz of PSDS.
Fig. \ref{fig: rampPmdif} shows mechanical power differences of less than $\pm$10 MW or, as Fig. \ref{fig: rampPmPercentdif} shows, 1\% difference max.
%*** Ramp results are included for now, but may be take up too much space... They do function as a `better' validation...

\begin{figure}[!ht]
	\centering
	\includegraphics[width=\linewidth]{figures/miniWECC3ALTDrampF3}
	\caption{Comparison of frequency during load ramp.}
	\label{fig: rampFcomp}
\end{figure}

\begin{figure}[!ht]
	\centering
	\includegraphics[width=\linewidth]{figures/miniWECC3ALTDrampRelF}
	\caption{Absolute difference of weighted frequency during load ramp.}
	\label{fig: rampFdif}
\end{figure}

\begin{figure}[!ht]
	\centering
	\includegraphics[width=\linewidth]{figures/miniWECC3ALTDrampPm2}
	\caption{Difference of mechanical power during load ramp.}
	\label{fig: rampPmdif}
\end{figure}

\begin{figure}[!ht]
	\centering
	\includegraphics[width=\linewidth]{figures/miniWECC3ALTDrampPm3}
	\caption{Percent difference of mechanical power during load ramp.}
	\label{fig: rampPmPercentdif}
\end{figure}
%-------------------------------------------------------------------------------------
\subsection{Validation Summary}
The validation results show that PSLTDSim accurately captures long-term power system dynamics. 
The step test results show that PSLTDSim cannot replace CTSS for the study of large, short-duration transient events.
However, repeated small perturbances, such as those associated with ramps, are modeled with acceptable deviation from CTSS methods.